\documentclass[20pt]{article}
\usepackage{geometry}
\geometry{letterpaper}
\usepackage[parfill]{parskip}    
\usepackage{graphicx}
\usepackage{amssymb}
\usepackage{epstopdf}
\usepackage{mathtools}
\usepackage{amsfonts}
\usepackage{amsmath}
\usepackage{amsthm}
\usepackage{verbatim}
\usepackage{amsfonts}
\usepackage{amscd}
\usepackage{graphicx}
\usepackage{mathrsfs}
\usepackage{graphics}
\usepackage{enumerate}
\usepackage{framed}
\usepackage[usenames, dvipsnames]{color}
\usepackage{hyperref}
\hypersetup{
    colorlinks,
    citecolor=black,
    filecolor=black,
    linkcolor=black,
    urlcolor=black
}




%Borders, spacing etc.
% \topmargin0.0cm
% \headheight0.0cm
% \headsep0.0cm
% \oddsidemargin0.0cm
% \textheight23.0cm
% \textwidth16.5cm
% \footskip1.0cm




\theoremstyle{plain}
\newtheorem{theorem}{Theorem}
\newtheorem{corollary}{Corollary}
\newtheorem{lemma}{Lemma}
\newtheorem{proposition}{Proposition}
\newtheorem*{surfacecor}{Corollary 1}
\newtheorem{conjecture}{Conjecture} 
\newtheorem{question}{Question} 
\theoremstyle{definition}
\newtheorem{definition}{Definition}
\newtheorem*{problem}{Problem}


%common sets
\newcommand{\reals}{\mathbb{R}}
\newcommand{\integers}{\mathbb{Z}}
\newcommand{\complex}{\mathbb{C}}
\newcommand{\rationals}{\mathbb{Q}}
\newcommand{\field}{\mathbb{F}}
\newcommand{\naturals}{\mathbb{N}}







\title{Integration Basics}
\date{\today} 
\author{Hugo Ludemann}













\begin{document}
\maketitle

\section{Comments}

There's a theme of `economy' in the presentation of the theorems.
For example, Fubini's Theorem is given in pretty intense generality, 
but several more specific results follow trivially from it, 
such as Leibnitz' Rule, equality of continuous mixed derivatives,
volumes of solids of revolution, Cavalieri's Principle and most importantly 
the volume of the image of a rectangle under a linear transformation.




















































\section{Issues}



These are incomplete/problematic:
\begin{itemize}
  \item 3.10 Would like a proof using the notions developed here rather than 
  topological facts.
  \item 3.11, 3.12, 3.19, 3.25, 3.28, 3.29, 3.31, 3.40, 3.41 Needs proof.
  \item Still not sure I understand what 3.21 is getting at.  Is it asking for a partition 
  which doesn't contain any rectangles in $A-C$? If so its easy to 
  draw JM sets with two connected components which seem to break this idea.
  \item 3.33 (b) straight up makes no sense. $f$ is defined on $[a, b] \times [c, d]$
  and $G(x)$ is defined, nonsensically, as $\int_a^{g(x)}f(t, x)dt$.
  \item Be more clear on the hypothesis of Fubinis for switching order in 3.35.
  \item 3-37 (b) is reportedly broken. Still needs doing with an improved 
  hypothesis.
  \item 4.4 might need a better comment on orientation.
\end{itemize}

It would be good to check whether an argument about absolute convergence should ever be made explicitly.  

In other proofs Lemma 3.1 is used generously, in the sense that if $(P')$ 
is a family of partitions subject to some condition which does not restrict the
`fine-ness' of $P'$
then each $P'$ is a refinement of some generic partition and so any result 
involving an infinumum/suprememum over all partitions holds equally well over
the restricted set of partitions.  Would be good to clear this up.

Personally I think the original statement 3.21 is pretty confusing.  It could 
state explicitly that the partition contains no subrectangles contained in $A-C.$
\vspace{3em}














































\section{Problems}

\begin{problem}{3.1}
  Let $f: [0,1] \times [0, 1] \to \reals$ be defined by
 \begin{equation*}
    f(x, y) =
    \begin{cases*}
      0 & if $0 \leq x < \frac{1}{2}$ \\
      1 & if $ \frac{1}{2} \leq x \leq 1 $.
    \end{cases*}
  \end{equation*}
Then $f$ is integrable and the value of the integral is $\frac{1}{2}.$
\end{problem}

\begin{proof}
  For $\varepsilon > 0$ let $P_\varepsilon$ be the partition with subrectangles 
  \begin{align*}
    S_1 &= [0, \frac{1}{2} - \frac{\varepsilon}{4}] \times [0, 1] \\
    S_2 &= [\frac{1}{2} -\frac{\varepsilon}{4}, \frac{1}{2} + \frac{\varepsilon}{4}] \times [0, 1] \\
    S_3 &= [\frac{1}{2} + \frac{\varepsilon}{4}, 1] \times [0, 1]
   \end{align*}
   Then 
   $U(f, P_\varepsilon) - L(f, P_\varepsilon)$ = $\dfrac{\varepsilon}{2} < \varepsilon$ 
   i.e. $f$ is integrable
   and $\int f = \inf U(f, P_\varepsilon) = \frac{1}{2}.$
\end{proof}


\begin{problem}{3.2}
  Let $f: A \to \reals$ be integrable and $g = f$ except at finitely many points.
  Then $g$ is integrable and $\int f = \int g.$
\end{problem}

\begin{proof}
  Suppose $f$ and $g$ differ except at a single point $x$.
  Let $P$ be some partition fine enough that $x$ lies in a single 
  subrectangle $S$. Then
  $$(U-L)(g, P) = (U-L)(f, P) + (M_S-m_S)(g-f)\cdot v(S)$$
  can be made arbitrarily small, as the term $(U-L)(f, P)$ vanishes by 
  integrability of $f$ and the other term can be made as small
  as we like by insiting that the rectangle $S$ is sufficiently small. 
  Furthermore if $g(x) > f(x)$ then 
  \begin{align*}
    \int g = \inf [U(g, P)] = \inf[ U(f, P) + (M_S(g - f)\cdot v(S))] = \inf[U(f, P)] = \int f
  \end{align*}
  with a similar argument for when $g(x) < f(x).$  
  Repetition of the argument proves the result for a finite collection of 
  points.
\end{proof}


\begin{problem}{3.3 Linearity of the Integral.}
  Let $f, g: A \to \reals$ be integrable.
  Then $\int f+g  = \int f + \int g$ and $\int cf =c\int f$.
\end{problem}

\begin{proof}
  Since 
  $$m_S(f) + m_S(g) \leq f(x) + g(x) \ \ \forall x \in S$$
  it follows that
  $$m_S(f) + m_S(g) \leq \inf_{S}(f + g) = m_S(f + g).$$
  Then $L(f, P) + L(g, P) \leq L(f + g, P)$ follows easily and by a similar argument 
  $U(f + g, P) \leq U(f, P) + U(g, P).$  Hence the differece 
  $(U-L)(f + g, P)$ is bounded from above by $(U-L)(f, P) + (U-L)(g, P)$ 
  which vanishes by integrability of $f$ and $g$.

  To compute the value of the integral, 
  \begin{align*}
    \int f + \int g = \sup L(f, P) + \sup L(g, P) = \sup[ L(f, P) + L(g, P) ] 
    \leq \sup L (f + g, P) = \int f + g \\
    \int f + \int g = \inf U(f, P) + \inf U(g, P) = \inf[ U(f, P) + U(f, P) ]
    \geq \inf U(f + g, P) = \int f + g
  \end{align*}
  and so $\inf f + \int g = \int f + g.$

  Finally for constants $c$, 
  \begin{align*}
    \inf U(cf, P) = c \inf U(f, P) = c \sup L(f, P) = \sup L(cf, P)
  \end{align*}
  i.e $cf$ is integrable with the value 
  $\int cf = \inf U(cf, P) = c \inf U(f, P) = c \int f.$
\end{proof}



\begin{problem}{3.4}
  Let $f: A\to \reals$ and let $P$ be a partition of $A$. Then $f$ is integrable
  if and only if the restriction of $f$ to each subrectangle is integrable, 
  and in this case $\int_A f = \sum_P \int_S f|_S.$
\end{problem}

\begin{proof}
  Let the $n$ subrectangles of $P$ be $S_i$ and suppose that each $f|_{S_i}$ is integrable.
  Then for any $\varepsilon > 0$ and each $i$ there is a partitions $P_i$ of $S_i$ such that 
  $(U - L)(f|_{S_i}, P_i) <\varepsilon / n.$
  Then 
  \begin{align*}
    \sum_{S_i \in P} \sum_{S_{ij} \in P_i} [ (M_{S_{ij}} - m_{S_{ij}})(f|S_i)\cdot v(S_{ij}) ] < 
    \varepsilon \Longrightarrow 
    \sum_{S_i \in P} \sum_{S_{ij} \in P_i} [ (M_{S_{ij}} - m_{S_{ij}})(f)\cdot v(S_{ij}) ] < 
    \varepsilon
  \end{align*}
  where the latter double sum is $(U - L)f$ over the partition formed by taking the union of
  all the $P_i.$  Hence $f$ is integrable.

  On the other hand for any rectangle $S \in P$  we have
  $$(U-L)(f|_S, S) = (U-L)(f, S) \leq (U - L)(f, P)$$
  since $U-L$ is non-negative and $S$ is containined in $P$. 
  If $f$ is integrable then the RHS can be made arbitrarily small by refining $P$, 
  in which case $S$ may need to be replaced by the part of that refinement which 
  covers it, $S = \cup P'$. Then $(U-L)(f|_S, P')$ vanishes and so $f|_S$
  is integrable.

  To compute the integral we have, for partitions $P_S$ of each $S$
  \begin{align*}
    \sum_P \int_{S} f|_S = 
    \sum_P \sup [L(f|_S, P_S)] = 
    \sup \bigg[\sum_P L(f|_S, P_s)\bigg] = 
    \sup \bigg[\sum_P \sum_{P_i} m_{S_ij}(f)\cdot v(S_{ij}) \bigg].
  \end{align*}
  The last term is the supremum over partitions $P'$ which are generic partitions
  of $A$ but for the refinement that each rectangle must be contained in one of the
  original $S \in P.$  
  Hence the supremum over such partitions coincides with the supremum over generic ones
  and so $\sum \int f|_S = \int f.$
\end{proof}


\begin{problem}{3.5}
  Let $g, f: A \to \reals$ be integrable and suppose $f \leq g$.  
  Then $\int f \leq \int g.$
\end{problem}

\begin{proof}
  \begin{align*}
  \int f \ =
  \ \ \inf \sum M_S(f) \cdot v(S)\ \  \leq 
  \ \ \inf \sum M_S(g) \cdot v(S) \ \ = 
  \ \int g.
  \end{align*}
\end{proof}


\begin{problem}
  If $f: A \to \reals$ is integrable then $|f|$ is integrable and
  $\big | \int f \big | \leq \int |f|.$
\end{problem}

\begin{proof}
  \begin{align*}
    (U-L)(|f|, P) = \sum (M_S(|f|) - m_S(|f|))\cdot v(S) 
    \leq \sum (M_S(f) - m_S(f))\cdot v(S)
  \end{align*}
  which is as small as we like by integrability of $f$. The required comparison
  follows from taking infinums of
  $$\big | \sum M_s(f) \cdot(S) \big |  \leq 
  \sum |M_s(f) \cdot(S)| = 
  \sum M_s(|f|) \cdot(S).$$ 
\end{proof}

\begin{problem}{3.7}
  Let $f: [0, 1]\times[0,1] \to \reals$ be defined by 
   \begin{equation*}
    f(x, y) =
    \begin{cases*}
      0 & if $x$ is irrational or y is irrational \\
      1/q & if $x$ is rational and $y=p/q$ in lowest terms.
    \end{cases*}
  \end{equation*}
  Then $f$ is integrable and the integral is $0.$
\end{problem}


\begin{proof}
  \color{ForestGreen}?
\end{proof}























































\begin{problem}{3.8}
  The rectangle $A = \Pi [a_i, b_i] \subset \reals^m$ 
  does not have content zero if each $a_i < b_i.$
\end{problem}

\begin{proof}
  Let $U_1, ..., U_n$ be a finite cover of closed rectangles with each $U_i$ contained in $A$ and
  let $P_d = (a_d = t_{d0} < ... < t_{dk(d)} = b_d)$ be the partition containing all 
  the endpoints of $U_i$ in the $d$th dimension, $d = 1,..,m$.
  Then each $v(U_i) = \Pi_{d=1}^m \sum ( t_{d,j} - t_{d,j-1} )$ for a 
  `certain number' of summands.
  Moreover each $\Pi_{d=1}^m(t_{d,j} - t_{d,j-1})$ lies in at least one $U_i,$
  so
  \begin{align*}
    \sum v(U_i) \geq 
    \prod_{d = 1}^m \sum_{j=1}^{k(d)}(t_{d,j} - t_{d,j-1}) = 
    \prod(b_d - a_d) = v(A)
  \end{align*}
  which is positive so long as each $a_i < b_i.$
\end{proof}

\begin{problem}{3.9}
  An unbounded set cannot have content zero, but can have measure zero.
\end{problem}

\begin{proof}
  If $A$ has content zero then there is a finite cover of $A$ by closed rectangles.
  Then each point of $A$ lies in some closed rectangle, i.e. $A$ is bounded.

  The set $\{1 ,2, 3, ...\}$ is clearly unbounded, but the cover 
   $n \in [n - \frac{\varepsilon}{2^{n+1}},n + \frac{\varepsilon}{2^{n+1}} ]$
   has volume $\varepsilon$, which is as small as we like.
\end{proof}

\begin{problem}{3.10}
  A bounded set of content zero has boundary of content zero, but 
  a bounded set of measure zero may have boundary with positive measure.
\end{problem}

\begin{proof}{Using results from topology.}
  Let $C$ be a bounded set, $\overline{C}$ the closure of $C$ and 
  $U_i$ a finite cover of $C$ by closed rectangles. Then $U = \cup U_i$ is a finite union 
  of closed sets, hence closed. 
  The closure of $C$ is by definition the smallest 
  closed set containing $C$.  Furthermore it can be shown that it also contains
  the bounday of $C$, whence
  $ \partial C \subset \overline{C} \subset U$ and so $\partial C$ has content zero.

  On the other hand the set $\rationals \cap [0, 1]$ is bounded and has zero measure
  but its boundary is that whole interval.
\end{proof}
\color{ForestGreen}
  Would be better to give a proof using only the theorems introduced in the book so far.
\color{Black}



\begin{problem}{3.11}
  Let $A \subset [0, 1]$ be the union over countably many intervals $(a_i, b_i)$
  such that each rational number in $(0, 1)$ is contianed in  some interval.
  Problem $1-18$ states that $\partial A = [0, 1] - A$.
  Show that if each $b_i > a_i$ then $A$ does not have zero measure.
\end{problem}
\begin{proof}
  \color{ForestGreen} Follows from the subaditivity of a measure, but since that tool has not been introduced need a different proof.
\end{proof}


\begin{problem}{3.12}
  Let $f: [a, b] \to \reals$ be increasing. Then the set 
  $X = \{ x: f\ discontinuous\ as\ f \}$ has measure zero.
\end{problem}

\begin{proof}
  \color{ForestGreen}
\end{proof}


\begin{problem}{3.13}
  Any open cover admits a countable subcover.
\end{problem}

\begin{proof}
  The first result is that the collection of `rational rectangles' - closed rectangles 
  with all rational endpoints - is countable. This is a consequence of theorem which says
  a finite cartesian product of countable sets is countable, after identifying the rectangle 
  in $\rationals^n$ with a point in $\rationals^{2n}$.

  For the main result, let $C_i$ be any open cover. For any $x$ in our set, 
  there is some $C_i$ containing $x$.  Then there is a closed rectangle $B$ with
  $x \in B \subset C_i$.  By a generalisation of the argument that any interval contains 
  infinitley many rational numbers, there is a rational rectangle $A_x$ with 
  $x \in A_x \subset B \subset C_i.$
  Then the union of all the $A_x$ is obviously a cover, which cannot be larger than $\cup_\naturals A_j$ since there are only countably many rational rectangles $A$.
  Then each $x$ is in some $A_j \subset C_j$, whence $C_j$ is a cover.
\end{proof}



























































\begin{problem}{3.14}
  If $f$ and $g$ are integrable then so is $f\cdot g$.
\end{problem}

\begin{proof}
  Let $D_f,\ D_g$ be the sets on which $f$ and $g$ are not continuous.
  Then (Theorem 3.8) both are measure zero and so (Theorem 3.4 / subaditivity of measure)
  $D_{f\cdot g} \subset D_f \cup D_g$ is measure zero.
  By Theorem 3.8 again $f \cdot g$ is integrable.
\end{proof}


\begin{problem}{3.15}
  If $C$ has content zero then $C$ is Jordan Measurable and 
  then volume of $C$ is zero.
\end{problem}

\begin{proof}
  Since $C$ is content zero it can be covered by finitely many closed rectangles.
  Let $a_i, b_i$ be the smallest and largest of the endpoints in each dimension,
  then define $A = \prod [a_i, b_i].$ Then $A$ contains $C$ which is therefore bounded.
  By problem 3.10 the boundary of $C$ has content zero, hence measure zero and so
  the $C$ is Jordan Measurable.

  To compute the integral take $\varepsilon > 0$ and let 
  $U_1, ..., U_n$ be some cover of $C$ which has volume smaller than $\varepsilon.$
  Let $P$ be some partition of $A$ consisting 
  of each $U$ together with other rectangles $V$.
  Then 
  \begin{align*}
    L(\chi_C, P) = \sum 1 \cdot v(U) + \sum 0 \cdot v(V) = \sum v(U) < \varepsilon
  \end{align*}
  and so $\int_C 1 = 0.$

\end{proof}

\begin{problem}{3.16}
  The same does not hold for bounded sets of measure zero.
\end{problem}

\begin{proof}
  Consider $\rationals \cap [0,1].$  The set is bounded and has zero measure but 
  (becasue the boundary is large) not integrable. 
  If $P$ is any partition, then each subrectangle $S$ contains both rational and
  irrational numbers, so $M_s(\chi_C) = 1$ and $m_s(\chi_C) = 0.$  Clearly the upper and lower 
  sums over $P$ can never agree.
\end{proof}

\begin{problem}{3.17}
  If $C$ is a bounded set of measure zero and $\int_C 1$ exists then it must be zero.
\end{problem}

\begin{proof}
  Let $A$ be some rectangle containing $C$ so that $\int_C 1 = \int_A \chi_C$
  and let $P$ be a partition of $A$.  Then 
  \begin{align*}
    L(\chi_C, P) = 
    \sum m(\chi_C, S)\cdot v(S) + \sum m(\chi_C, Q) \cdot v(Q) = 
    \sum v(S)
  \end{align*}
  where $S$ are the subrectangles contained in $C$ and $Q$ are the rest.
  The inclusion $S \subset C$ implies $S$ is also zero measure, and 
  so by problem 3.8 is a degenerate rectangle. This is not allowed in 
  our definition of a partition and so all subrectangles 
  are of the type $Q$, so $L(\chi_C, P) = 0$ for any partition $P$.
  We have assumed that the integral exists, and so its value must be zero.
\end{proof}


\begin{problem}{3.18}
  If $f: A\to\reals$ is non-negative and $\int f = 0$ then the
  support of $f$ has measure zero.
\end{problem}

\begin{proof}
  For any integer $n$ and $\varepsilon > 0$ let $P$ be a partition with 
  $\sum M_S(f)\cdot v(S) < \varepsilon/n.$ 
  Then either $M_S(f) > 1/n$ or $M_S(f) \leq 1/n$ and so 
  \begin{align*}
    \sum_{M\leq 1/n} M_S(f) \cdot v(S) + \sum_{M > 1/n} \frac{1}{n}v(S) < 
    \varepsilon/n
  \end{align*}
  from which (since both sums are finite)
  \begin{align*}
    \sum_{i = 1,...k} v(S_i) < \varepsilon.
  \end{align*}

  If $D_n$ is defined as  $\{ x: f(x) > 1 / n \}$ then it is surely covered by these
  $S$, which the argument above shows to be a cover of content zero.
  The support of $supp(f) = D_1 \cup D_2 \cup ...$ is a countable union of 
  measure zero sets, hence measure zero.
\end{proof}


\begin{problem}{3.19}
  Let $U$ be the open set in problem 3.11.
  If $f = \chi_U$ except on a set of measure zero then $f$ is not integrable.
\end{problem}

\begin{proof}
  \color{ForestGreen}?
\end{proof}


\begin{problem}{3.20}
  An increasing function from $[a, b]$ to $\reals$ is integrable.
\end{problem}
\begin{proof}
  By problem 3.12 the subset of the domain where the function is 
  discontinuous has measure zero. The function is therefore integrable by theorem 3.8.
\end{proof}


\begin{problem}{3.21}
  Let $A$ be a closed rectangle and $C \subset A$ and let $P$ be some 
  partition in which every subrectangle is either contained in $C$, 
  or intersects $C$ (without being contained.)  That is, there are no
  subrectangles in $A-C$.
\end{problem}

\begin{proof}
\color{ForestGreen} ?
\end{proof}

\begin{problem}{3.22}
  If $A$ is a JM set and $\varepsilon > 0$ there is a compact JM set $C \subset A$
  such that $\int_{A-C}1 < \varepsilon.$ 
\end{problem}

% \begin{proof}
%   Let $U_1, ...U_n$ be a cover of $\partial A$ by \textit{open} rectangles and define
%   \begin{align*}
%     C = \overline{A} - \bigcup U_i = \overline{A} \cap \big (\bigcup U_i^c\big )
%   \end{align*}
%   where $X^c$ denotes the complement of $X$ in the ambient space, $X-C$
%   and $\overline{A}$ is the closure of $A$.  $C$ is compact so it remains to show that
%   it has a small boundary.  

%   If $x$ is some point in $\partial C$ then any neighbourhood of $x$
%   contains a point $a \in C$, which the definition of $C$ directly
%   implies is not in any of the $U_i$.
%   The same neighbourhood contains another point $b$ which is not in $C$, i.e. 
%   \begin{align*}
%     b \in \overline A^c  \bigcup U_i.
%   \end{align*}
%   If $b$ is not in $\overline{A}^c$ then it must be in a $U_i$.
%   Otherwise it is in $\overline{A}^c$ ie. is in the exterior of $A$. Hence our 
%   rectangle contains a point in the exterior of $A$ (that one is $b$) and a point in 
%   the closure of $A$ (that one is $x$).
%   \footnote{since $C\subset A \Rightarrow \partial C \subset \overline{A}.$}
%   Then it contains a point in 
% \end{proof}

\begin{proof}
  \color{ForestGreen}
  Baffling statement.
\end{proof}






















































































































\begin{problem}{3.23}
  Let $C = A \times B$ be a set of content zero.
  Let $A' \subset A$ be the set of $x \in A$ such that 
  $\{ y \in B: (x, y) \in C \}$ is not content zero.
  Then $A'$ is measure zero.
\end{problem}
\begin{proof}
  By problem 3.15 $C$ is Jordan Measurable and has volume $0$.
\end{proof}



\begin{problem}{3.24}
  The result of problem 3.23 does not hold in general for sets of measure zero.
\end{problem}

\begin{proof}
  Let $C\subset [0,1] \times [0,1]$ be the union of all 
  $\{ p/q \} \times [0, 1]$ where $p/q$ is a rational number in $[0, 1]$ with 
  $p/q$ in lowest terms. It's a werid comb kind of thing.
  The $C$ has content zero as the bounded countable union of measure zero sets, but 
  for any $p/q$ the set $\{ y \in [0, 1]: (p/q, y) \in C \} = [0, p/q]$
  is not content zero for any $p/q$, i.e. $A'$ is all rationals in $[0,1]$, which 
  is not content zero.
\end{proof}



\begin{problem}{3.25}
  Non-degenerate rectangles have positive measure.
  (Alternative proof to problem 3.8.  Useful, because that proof sucks.)
\end{problem}

\begin{proof}
  \color{ForestGreen}
  Seems to be more going on than I thought.
\end{proof}
  


\begin{problem}{3.26}
  Let $f: [a, b] \to \reals$ be integrable and non-negative and let 
  $A_f$ be the region between the graph of $f(x)$ and the $x$-axis.
  Then $A_f$ is Jordan Measurable and has area $\int_a^b f.$
\end{problem}

\begin{proof}
  The boundary of $A$ consists of three bounded line segments and the graph of $f$.
  Since $f$ is integrable, it is continuous almost everywhere and so the 
  the graph is a countable union\footnote{A measure zero subset of an interval must be a countable union of singletons.}
   of continuous plane curves.  Together with 
  the obvious fact that $A_f$ is bounded establishes that $A_f$ is 
  Jordan measurable.

  Let $M$ be some upper bound for $f$.  Then 
  \begin{align*}
    \mathcal{L}(x) = \int_{[0, M]} \chi_{A_f}(x, y)dy = f(x)
  \end{align*}
  and so by Fubini the volume of $A_f$ is 
  \begin{align*}
    \int_{A_f}1 = 
    \int_{[a, b]\times[0, M]} \chi_{A_f}
    \int_{[a, b]} \mathcal{L}(x) = 
    \int_a^b f.
  \end{align*}
\end{proof}



\begin{problem}{3.27}
  If $f: [a, b] \times [a, b] \to \reals$ is continuous then 
  \begin{align*}
    \int_a^b \int_a^y f(x, y)dxdy = \int_a^b \int_x^b f(x, y)dy  dx.
  \end{align*}
\end{problem}

\begin{proof}
  Let $C$ be the region above the diagonal, that is 
  \begin{align*}
    \{ (x, y) : y \in [a, b]\ and \ x \in [a, y] \} = 
    \{ (x, y) : x \in [a, b]\ and \ y \in [x, b] \}.
  \end{align*}
  Then 
  \begin{align*}
    \int_C f = 
    \int_{[a, b], [a, b]}\chi_C \cdot f = 
    \int_{[a, b]}\bigg(\int_{[a, b]}(f\cdot \chi_C)(x, y)dy\bigg)dx = 
    \int_a^b \bigg(\int_x^b f dy\bigg)dx
  \end{align*}
  with the other side following from a similar calculation.
\end{proof}


\begin{problem}{3.28}
  $D_{1, 2}f = D_{2, 1}f$ so long as these are continuous.
\end{problem}
\begin{proof}
  \color{ForestGreen}?
\end{proof}




\begin{problem}{3.29}
  Find the volume of the solid obtained by rotating a 
  JM set in the $yz$ plane about the $z$-axis.
\end{problem}
\begin{proof}
  \color{ForestGreen}?
\end{proof}





\begin{problem}{3.30}
  Let $C$ be the set in problem $1.17$ (the evil dense subset of the unit square 
  containing at most one point on any horizontal or vertical line.)
  Then both $\int_0^1 \int_0^1 \chi_C dydx$ and  $\int_0^1 \int_0^1 \chi_C dxdy$
  are zero but the integral over $C$ does not exist.

  In other words, the converse of Fubini's theorem does not hold in general.
\end{problem}

\begin{proof}
  For any fixed $x$, the integral $\int_0^1\chi_C(x, y)dy$ is clearly zero since
  $C$ contains at most a single point on the line $\{x\} \times [0,1]$ from which
  it follows that the first double integral is $0$.  The argument for the other one
  is exactly the same.

  Since $C$ is dense, any rectangle contains both a point inside $C$ and
  a point outside $C$ and so $M_S = 1, m_S = 0$ for any rectangle $S$.
  Clearly the upper and lower integrals cannot coincide.
\end{proof}


\begin{problem}{3.31}
  For $A = [a_1, b_1] \times ...\times [a_n, b_n]$ and $f: A \to \reals$
  define $F: A: \reals$ by
  \begin{align*}
    F(x) = \int_{[a_1, b_1]\times ... \times[a_n, b_n]}f.
  \end{align*}
  What is $D_if(x)$ for $x$ in the interior of $A$?
\end{problem}

\begin{proof}
  \color{ForestGreen}?
\end{proof}



\begin{problem}{3.32: Leibnitz' Rule}
  Let $f: [a, b] \times [c, d] \to \reals$ be continuous and suppose
  $D_2f$ is continuous.  Define $F(y) = \int_a^bf(x, y)dx.$
  Then $F'(y) = \int_a^bD_2f(x, y)dx.$
\end{problem}

\begin{proof}
  Rewrite $F(y)$ as 
  \begin{align*}
    \int_a^b \int_c^y D_2f(x, t)dtdx + \int_a^b f(x, c) dx.
  \end{align*}
  By continuity of $D_2f$ the order of integration may be reversed.
  The result follows from the fundamental theorem of calculus applied to the
  first integral and the $y$-independence of the second.
\end{proof}


\begin{problem}{3.33}
  If $f:[a, b] \times [c, d] \to \reals$ is continuous, define
  $F(x, y) = \int_a^xf(t, y)dt.$ Find $D_1F$ and $D_2F$.
\end{problem}

\begin{proof}
  By the FTC, $D_1F(x, y) = f(x, y)$ and by Leibnitz' rule $D_2F(x, y) = \int_a^bD_2f(t, y)dt.$
\end{proof}





\begin{problem}{3.34}
  Let $g_1, g_2: \reals^2 \to \reals$ be continuously differentiable and 
  suppose $D_1g_2 = D_2g_1.$  Define 
  \begin{align*}
    f(x, y) = \int_0^x g_1(t, 0)dt + \int_0^yg_2(x, t)dt.
  \end{align*}
  Then $D_1f(x, y) = g_1(x, y).$
\end{problem}

\begin{proof}
  By FTC and Leibnitz' rule
  \begin{align*}
    D_1f(x, y)
    &= g_1(x, 0) + \int_0^yD_1g_2(x, t)dt \\
    &= g_1(x, 0) + \int_0^yD_2g_1(x, t)dt \\
    &= g_1(x, 0) + g_1(x, y) - g_1(x, 0) \\
    &= g_1(x, y).
  \end{align*}
\end{proof}



\begin{problem}{3.35}
  If $G: \reals^n \to \reals^n$ is a invertible linear transformation and $U$ a 
  rectangle, then the volume of $G(U)$ is $|\det G|\cdot v(U).$
\end{problem}

\begin{proof}
  Let $U = [a_1, b_1] \times ... \times [a_n, b_n]$
  and consider first of all three kinds of elementary transformations.  
  \begin{enumerate}
     \item $G$ takes one of the basis vectors $e^i$ to $ce^i$ and fixes the rest.
     If $c$ is positive then 
     \begin{align*}
       v(G(U))
       &= v([a_1, b_1]\times...\times[ca_i, c_bi]\times...\times[a_n, b_n]) \\
       &= (b_1 - a_1)\cdot...\cdot(cb_i - ca_i)\cdot ... \cdot(b_n-a_n) \\
       &= c\cdot v(U).
     \end{align*}
     The matrix representation of $G$ is the identity with the $i$-th 
     entry replaced by $c$, which has determinant $c.$ 
     If $c$ is negative then the absolute value signs are needed but 
     the volume is unchanged, although $[a, b]$ must become $[cb, ca]$. 

     \item $G$ swaps two dimensions and fixes the rest, i.e. 
     $g(e_i) = e_j$ and $g(e_j) =e_i.$
     As a matrix $G$ is the identity with zeros in the $i$ and $j$-th positions, 
     and ones in the $G_{i, j}$ and $G_{j,i}$ positions.  In $\reals^3$ it might look like
     \begin{align*}       
       \begin{pmatrix}
         1 & 0 & 0 \\
         0 & 0 & 1 \\
         0 & 1 & 0
       \end{pmatrix}
     \end{align*}
     and in any dimension larger than one has determinant $-1.$  
     Clearly this operation has no effect on the volume of the rectangle 

     \item $G$ adds basis vector to another and fixes the rest, i.e.
     \begin{align*}
       G(x^1,..., x^i,..., x^j,..., x^n) = (x^1,..., x^i + x^j,..., x^j,..., x^n).
     \end{align*}
     As a matrix $G$ is the identity with a single extra one somewhere, and so it is easy to 
     see that $\det G = 1$.


     Linear operators on $\mathbb{R}^n$ are bounded, so the volume of $G(U)$
     can be calculated as 
     \begin{align*}
       \int_{[-M_n, M_n]\times...\times [-M_1,M_1]}\chi_{G(U)}dx^1...dx^n
     \end{align*}
     for some upper bound $M$ of both $U$ and $G(U)$.  Since Fubini's theorem will allow us to interchange
     the order of integration, we can assume that $G$ 
     adds $e_2$ to $e_1$ and leaves the rest. Then $v(G(U)) = $
     \begin{align*}
       \int_{[-M_n, M_n]\times...\times[-M_3, M_3]} \bigg( 
          \int_{[-M_2,M_2]\times[-M_1,M_1]} \chi_{G(U)}(x^1, x^2, ...)dx^1dx^2
        \bigg) dx^3...dx^n.
     \end{align*}
     The value of the inner integral is the area of a parallelogram, which 
     coincides with the area of the rectangle which fell over.  Hence this inner integral 
     would not be changed by replacing $\chi_{G(U)}$ with $\chi_U.$
     Furthermore, for fixed $a^1$ and $a^2$ and varying coordinates in the 
     remaining positions, $\chi_U(a_1, a_2, x_3, ..., x_n)$ and 
     $\chi_{G(U)}(a_1, a_2, x_3, ..., x_n)$ conincide exactly.
     So for the purposes of evaulating this integral, $\chi_U$ will do 
     just as well as $\chi_{U(G)}$, or in other words their volumes are equal.
     
     \end{enumerate}

     Finally if $G$ is the composition of elementary operations as described above
     then $G$ alters the volume of $U$ by a factor of 
     \begin{align*}
        |\det G_1| \cdot .. \cdot |\det G_m| = 
        |\det G_1 \cdot ... \cdot \det G_m | = 
        |\det (G_1 \cdot ... \cdot G_m) | = |\det G|.
     \end{align*}
     We take as a fact that these elemetary operations generate the whole group 
     of invertible matrices and so this scaling factor is good for any invertible linear map.
\end{proof}




\begin{problem}{3.36 Cavalieri's Principle.}
    Let $A$ and $B$ be JM subsets of $\reals^3$. 
    Let $A_c = \{(x, y): (x, y, c) \in A\}$ and define $B_c$ similarly.
    If each $A_c$ and $B_c$ are JM  and have the same area then $A$ and $B$
    have the same volume.
\end{problem}

\begin{proof}
  Let $[-M, M]^3$ be a rectangle bounding $A$ and $B$. Then 
  \begin{align}
    \int_{[-M, M]^3}\chi_A 
    &= \int_{-M}^M \bigg( \int_{[-M, M]^2} \chi_A \bigg)dz \\
    &= \int_{-M}^M \bigg( \int_{A_z} 1 \bigg) dz
    = \int_{-M}^M \bigg( \int_{B_z} 1 \bigg) dz \\
    &= \int_{-M}^M \bigg( \int_{[-M, M]^2} \chi_B \bigg)dz 
    = \int_{[-M, M]^3}\chi_B.
  \end{align}
\end{proof}







































































































































































\begin{problem}{3.37}
  Suppose that $f: (0, 1) \to \reals$ is non-negative and continuous.
  Then $\int_{(0,1)}f$ exists if and only if 
  $\lim_{\varepsilon \to 0} \int_\varepsilon^{1-\varepsilon}f$ exists.
\end{problem}

\begin{proof}
  Let $o$ be the cover of $(0, 1)$ consisting of the intervals
  $(1/n, 1 - 1/n)$ and $\Phi$ a $C^\infty$ partition of unity for $(0, 1)$
  subordinate to $o$.
  Then each $\phi_i$ in $\Phi$ has support contained in some 
  $(1/m_i, 1 - 1/m_i)$ in $o$.
  Choose an ordering of $\Phi$ so that if $i < j$ then $m_i \leq m_j$, that way
  the support of $\phi_i$ is in $ (1 / m_j, 1 - 1/m_j) $ for all $j > i.$

  Then the partial sums can be written as
  \begin{align*}
    \sum_{k=1}^n \int_{(0, 1)} \phi_k \cdot f = 
    \sum_{k=1}^n \int_{1/m_k}^{1 - 1/m_k} \phi_k \cdot f = 
    \sum_{k=1}^n \int_{1/m_n}^{1 - 1/m_n} \phi_k \cdot f =  
    \int_{1/m_n}^{1 - 1/m_n} \sum_{k=1}^n  \phi_k \cdot f =  
    \int_{1/m_n}^{1 - 1/m_n} f.
  \end{align*}
  Since $\phi \cdot f$ is non-negative, convergence and absolute conergence
  are the same thing. Thus the extended integral exists if and only if the 
  given limit exists.
\end{proof}


\begin{problem}{3.38}
  Let $A_n$ be a closed set in $(n, n+1)$. Suppose that $f: \reals \to \reals$
  satisfies $\int_{A_n}f = (-1)^n/n$ and $f=0$ outside $A_n.$ 
  Then $f$ is not integrable.
  % Find two different partitions of unity for $\mathbb{R}$ for which 
  % the series $\sum_{\Phi} \int_{A_n} \phi \cdot f$ converge absolutely
  % to different values.
\end{problem}

\begin{proof}
  Let $\Phi_n$ be a partition of unity for $A_n$ and extend each 
  $\phi$ to a function on $\mathbb{R}$ by setting $\phi=0$ outside 
  $A_n$.
  Then their union $\Phi$ is a partion of unity for all of $\mathbb{R}$
  and 
  \begin{align}
    \sum_\Phi \int_\mathbb{R} \phi \cdot |f| = 
    \sum_\naturals \bigg( \sum_{\Phi_n} \int_\reals \phi \cdot |f| \bigg) = 
    \sum_\naturals \bigg( \sum_{\Phi_n} \int_{A_n} \phi \cdot |f| \bigg) = 
    \sum_\naturals \int_{A_n}f = -\log 2.
  \end{align}

  Following the example in the wiki for conditional convergence, 
  define the sets 
  \begin{align}
    B_n = A_{2n - 1} \cup A_{2(2n-1)} \cup A_{4n}
  \end{align}
  and let $\Phi_n$ be a partition of unity for $B_n$, taking the 
  value $0$ outside $B_n$.  Then as above
  \begin{align}
    \sum_\Phi \int_\reals \phi \cdot |f| = 
    \sum_\naturals \bigg(\sum_{\Phi_n} \int_\reals \phi \cdot |f| \bigg) = 
    \sum_\naturals \bigg(\sum_{\Phi_n} \int_{B_n} \phi \cdot |f| \bigg) = 
    \sum_\naturals \int_{B_n}f.
  \end{align} 
  Each of the integrals in the last sum has the value 
  \begin{align}
    \int_{B_n}f
    &= \int_{A_{2n-1}}f + \int_{A_{2(2n-1)}}f + \int_{A_{4n}}f \\
    &=  \frac{(-1)^{2n-1}}{2n-1} + 
        \frac{(-1)^{2(2n-1)}}{2(2n-1)} + 
        \frac{(-1)^{4n}}{4n} \\
    &=  \frac{1}{2(2n-1)} +
        \frac{1}{4n} -
        \frac{1}{2n-1} \\
    &= \frac{1}{2} \bigg(
     \frac{-1}{2n-1} + \frac{1}{2n} \bigg)
  \end{align}
  and so the sum converges to $-1/2 \log 2.$
\end{proof}













































































































































\begin{problem}{3.39}
  The condition that $\det g$ in the formula for change of variables is
  superfluous, due to Sard's Theorem.
\end{problem}
\begin{proof}
  \color{ForestGreen} ?
\end{proof}



\begin{problem}{3-40}
  If $g: \reals^n \to \reals^n$ and $\det g'(x) \neq 0$ then 
  there is an open set containing $x$ in which 
  $g = T \circ g_n \circ ... \circ g_1$ where each $g_i$
  is of the form $g_i(x) = (x^i, ..., f_i(x^i), ..., x^n)$
  and $T$ is a linear transformation.  Furthermore $g$ can be written as $g = g_n \circ...\circ g_1$ if and only if $g'$ is 
  diagonal.
\end{problem}

\begin{proof}
  \color{ForestGreen} ?
\end{proof}

\begin{problem}
  \begin{align}
    \int_{-\infty}^\infty e^{-x^2}dx = \sqrt{\pi.}
  \end{align}
\end{problem}

\begin{proof}
  \begin{enumerate}
    \item Define $f: \{r: r > 0 \} \times (0, 2\pi) \to \reals^2$ by
    $f(r, \vartheta) = r(\cos \vartheta, \sin \vartheta).$
    If $r(\cos \vartheta, \sin \vartheta) = r'(\cos \vartheta', \sin \vartheta')$
    then the Pythagorean identity easily gives $r = r'$, and so 
    $$\cos \vartheta = \cos \vartheta',\ \ \  \sin \vartheta = \sin \vartheta'.$$




  \end{enumerate}
\end{proof}





















\break
\section{Integration on Chains}



\begin{problem}{4.1}
  Let $e_1, ..., e_n$ be the usual basis of $\mathbb{R}^n$ and let $\phi_i$ be the dual basis.
  Then \\
  $(a) \ \phi_{i_1} \wedge ... \wedge \phi_{i_k} (e_{i_1}, e_{i_k}) = 1$ and
  $(b)$ $\phi_{i_1} \wedge ... \wedge \phi_{i_k}(v_1, ..., v_k)$ is the determinant of 
  the $k \times k$ minor of 
  $$\begin{pmatrix}v_1 \\ ...\\ v_k\end{pmatrix}$$
  obtained by selecting $i_1, ..., i_k$.
\end{problem}

\begin{proof}
  By induction on $k$.
  If $k = 1$ then $\phi_i(e_i) = 1.$
  Assume the result for $k-1$. Then
  \begin{align*}
    (\phi_{i_{1}} \wedge ... \wedge \phi_{i_{k-1}}) \wedge \phi_{i_{k}}
    (e_{i_1}, ..., e_{i_{k-1}}, e_{i_{k}})=
    \frac{1}{(k-1)!} \sum_{S_k}sign(\sigma)(\phi_{i_{1}} \wedge ... \wedge \phi_{i_{k-1}})(e_{\sigma(i_1)}, ..., e_{\sigma(i_{k-1})}) \cdot
     \phi_{i_{k}}(e_{\sigma(i_k)}).
  \end{align*}

  Since $\phi_{i_{k}}(e_{\sigma(i_k)})$ is one if $\sigma$ fixes $i_k$ and zero otherwise and
  $\phi_{i_{1}} \wedge ... \wedge \phi_{i_{k-1}}$ is alternating this reduces to 
  \begin{align*}
    \frac{1}{(k-1)!} \sum_{S_{k-1}}sign(\sigma)(\phi_{i_{1}} \wedge ... \wedge \phi_{i_{k-1}})(e_{\sigma(i_1)}, ..., e_{\sigma(i_{k-1})}) = \\
    \frac{1}{(k-1)!} \sum_{S_{k-1}}sign(\sigma)^2(\phi_{i_{1}} \wedge ... \wedge \phi_{i_{k-1}})(e_{i_1}, ..., e_{i_{k-1}}) = \\
    \frac{|S_{k-1}|}{(k-1)!} \cdot  \phi_{i_{1}} \wedge ... \wedge \phi_{i_{k-1}} (e_{i_1}, ..., e_{i_{k-1}}) = 1
  \end{align*}
  by the inductive hypothesis.

  \color{ForestGreen} Part (b) ?
\end{proof}


  \begin{problem}{4.2}
    Let $f: V \to V$ be linear and $\dim V = n$.
    Then $f^*: \Omega^n(V) \to \Omega^n(V)$ is multiplication by $\det f$.
  \end{problem}

  \begin{proof}
    % Let $v_1, ..., v_n$ be a basis of $V$, and $w_1, ..., w_n$
    \color{ForestGreen}?
  \end{proof}


  \begin{problem}{4.3}
    If $\omega$ is the volume element determined by $T$ and $\mu$ and $w_1, ..., w_n$ are vectors in $V$ then 
    \begin{align}
      |\omega(w_1, ..., w_n)| = \sqrt{det(g_{ij})}.
    \end{align}    
    Note that the hint for the problem has the indicies around the wrong way.
  \end{problem}
  \begin{proof}
    Let $v_1, ...v_n$ be an orthonormal basis with resepect to $T$ and $A = a_{ih}$ numbers such that 
    $w_i = \sum_{h=1}^n a_{ih}v_h$.
    Then 
    \begin{align}
      g_{ij} := T(w_i, w_j) = \sum_{h,g=1}^n a_{ih}a_{jg}T(v_i, v_j) = \sum_{k=1}^n a_{ik}a_{jk},
    \end{align}
    that is $(g_{ij}) = A\cdot A^T$ and
    $$\sqrt{\det G} = \det A.$$

    By theorem 4.6 we have 
    $ \omega(w_1, ..., w_n) = \det A \cdot \omega(v_1, ...v_n) = \pm \det(A)$
    (since $\omega$ is the volume element with respect to the orthonormal basis $v_i$) which proves the result.
  \end{proof}












































\begin{problem}{4.4}
  If $\omega$ is the volume element of $V$ determined by $T$ and $\mu$ and
  $f: \reals^n \to V$ is an isomorphism such that $f^*T = \langle, \rangle$ and such that 
  $[f(e_1), ..., f(e_n)] = \mu$ then $f*\omega = \det.$
\end{problem}

\begin{proof}
  If $(f(e_i))$ were an orthonormal basis with positive orientation then 
  \begin{align*}
    1 = \omega( f(e_1), ..., f(e_n) ) = f*\omega(e_1, ..., e_n)
  \end{align*}
  i.e. $f^*\omega$ is the volume element in $\reals^n$.  Since this is unique it equals $\det$.

  Orthonormailty of $(f(e_i))$ follows from the easy calculation
  \begin{align*}
    T(f(e_i), f(e_j)) = f^*T(e_i, e_j) = \langle e_i, e_j \rangle = \delta{ij}.
  \end{align*}
\end{proof}
















\begin{problem}{4.5}
    If $c: [0, 1] \to (\reals^n)^n$ is continuous and each $(c^1(t), ..., c^n(t))$ is a basis, 
    then any two have the same orientation. 
\end{problem}

\begin{proof}
  Since $c(t)$ is a basis for any $t$ it must always be non-singular.
  Hence $\det \circ c \neq 0$ on $[0, 1]$, which means it is always 
  positive, or always negative.
  Let $C_1$ and $C_2$ be the matrices of two bases in this family, and $A$ the transformation between them. 
  Then 
  \begin{align*}
    \det A = \det(C^{-1}_1) \cdot \det C_2 > 0
  \end{align*}
  as the argument above ensures the determinants on the right have the same sign.
\end{proof}










\begin{problem}{4.8}
  If $\omega \in \Omega^n(V)$ is a volume element define a `cross product' in terms of $\omega$.
\end{problem}
% \begin{prvf}










































\break



\section{Definitions}

\begin{theorem}
  Let $A \subset \reals^n$ and $o$ an open cover of $A$.
  Then there is countable collection $\Phi$ of $C^\infty$ 
  functions $\phi$ defined on $A$ with the following properties:
  \begin{enumerate}
    \item For any $x \in A$, $0 \leq \phi(x) \leq 1$.
    \item For any $x \in A$ there is an open set $V$ containing $x$ 
    such that all but finitely many $\phi$ are zero on $V$.
    \item For each $x \in A$, we have $\sum_\Phi \phi(x) = 1$ and 
    by (2) this sum is finite in some open set around $x$.
    \item For each $\phi \in \Phi$ there is an open set $U \in o$
    such that $\phi = 0$ outisde some closed set contained in $U$ 
    (the support of each $\phi$ is compact and contained in some 
    $U \in o$.)
  \end{enumerate}

  A colletion satisfying (1) - (3) is called a $C^\infty$ partition of unity for $A$.
  If it also satisfies (4) it is said to be subordinate to the cover $o$.
\end{theorem}

\vspace{2em}


\begin{definition}
  Let $o$ be an admissible open cover of an open set $A \subset \mathbb{R}^n$
  (each $U \in o$ is contained in $A$) and $\Phi$ a partition of unity 
  subordinate to $o$.  If $f: A \to \reals$ is bounded in some 
  open set around each point in $A$ and the subset of $A$ where $f$
  is discontinuous is measure zero then, each 
  $\int_A \phi \cdot f$ exists.
  If the series $\sum_\Phi \int_A \phi \cdot |f|$ converges 
  then we say $f$ is integrable write 
  $\int_A f = \sum_\Phi \int_A \phi \cdot f$.
\end{definition}

\vspace{3em}
\begin{definition}
  A $k-tensor$ is a multilinear map function from 
  $V^k \to \reals$ where $V$ is some vector space.
  The set of all $k$-tensors is denoted $\mathfrak{T}^k(V)$
  and becomes a is a vector space over $\mathbb{R}$.
  If $S \in \mathfrak{T}^k(V)$ and $T \in \mathfrak{T}^l(V)$
  define a tensor-product:
  \begin{align}
    S \otimes T (v_1, ..., v_k, v_{k+1}, ..., v_{k+l}) = 
    S(v_1, ..., v_k)\cdot T(v_{k+1}, ..., v_{k+l}).
  \end{align}
  This product is left and right distributive and associative.
\end{definition}

\vspace{3em}
\begin{theorem}
  Let $v_1, ..., v_n$ be a basis for $V$, and let 
  $\varphi_1, ..., \varphi_n$ be the dual basis, $\varphi_i(v_j) = \delta_{ij}.$
  Then the set of all 
  \begin{align}
    \phi_{i_1} \otimes ... \otimes \phi_{i_k}, \ \ 1 \leq i_1 \leq ... \leq i_k \leq n
  \end{align}
  is a basis for $\mathfrak{T}^k(V)$, which therefore has dimension $n^k.$
\end{theorem}


\begin{definition}
  A $k$-tensor is alternating if
  \begin{align}
    a(v_1, ..., v_i, v_j, ..., v_k) = -a(v_1, ..., v_j, v_i, ..., v_k)
  \end{align}
  i.e. two elements are swapped and the rest left alone.
\end{definition}



\vspace{3em}
\begin{definition}
  If $T$ is some $k$-tensor, 
  then you can have an alternating tensor for free:
  \begin{align}
    Alt(T) = \frac{1}{k!}\sum_{\sigma \in S_k} sign(\sigma)
      T(v_{\sigma1}, ..., v_{\sigma k})
  \end{align}
  where $S_k$ is the group of permutations on $k$ letters.
\end{definition}

\vspace{3em}
In general, the tensor product of two alternating tensors is not likely to be alternating.  Enter the wedge:


\begin{definition}
  Let $\omega \in \Omega^k(V)$ and $\nu \in \Omega^l(V)$.
  Then
  \begin{align}
    \omega \wedge \nu = \frac{(k+l)!}{k!l!}Alt(\omega \otimes \nu)
  \end{align}
  is in $\Omega^{k+l}(V).$
\end{definition}





\end{document}  